%!TEX TS-program = xelatex
\documentclass[11pt, print]{friggeri-cv}
\addbibresource{bibliography.bib}
\begin{document}
\header{Zhen-Huan Hu}
       {Biostatistician}

% In the aside, each new line forces a line break
\begin{aside}
  \section{Contact}
    3125 Fountain Sq Blvd
    Apt 203
    New Berlin
    53151 WI
    ~
    \href{mailto:zhu@mcw.edu}{zhu@mcw.edu}
    Phone: (608)-774-3021
  \section{Languages}
    English: proficient
    Mandarin: native
    Japanese: conversant
  \section{Programming}
    SAS, R, SQL
    Python, Perl, C
    Linux Shell Script
\end{aside}

Biostatistician, data analyst, and SAS programmer in a wide variety of business applications. Specialize in analyzing large observational databases. Particularly interested in pharmaceutical, health insurance and health consulting industry.\\

\section{Professional Experience}

\begin{entrylist}
  \entry
    {since 2011}
    {Biostatistician - Medical College of Wisconsin}
    {Milwaukee, WI}
    {
    \textit{Center for International Blood and Marrow Transplant Research}.
    \begin{itemize}
      \item Administrated 5-10 observational studies each year to ensure their timely and successful publications on internationally recognized peer-reviewed scientific journals.
      \item Provided integrated statistical solutions for physician scientists by generate highly sophisticated statistical reports using advanced statistical packages (SAS, R), database applications (Oracle SQL Developer, Perl), and graphic processing software.
      \item Presented annual statistical reports for pharmaceutical companies, insurance companies and non-profit scientific organizations, such as \textit{Blue Cross Blue Shield}, and the \textit{Worldwide Network for Blood \& Marrow Transplantation}.
      \item In charge of training entry-level statisticians in familiarizing the use of databases and SAS macro libraries.
    \end{itemize}
    }
  \entry
    {2009-2011}
    {Epidemiologist Intern - Minnesota Department of Health}
    {St. Paul, MN}
    {
    \textit{Center for Disease Prevention and Health Promotion}.
    \begin{itemize}
      \item Participated in \textit{I Can Prevent Diabetes} program, a community-driven project with 4 years of follow-up surveys. Provided statistical analyses on the survey feedback to evaluate the efficacy of the prevention program.
      \item Prepared multiple scientific reports for successful grant applications and conference presentations.
    \end{itemize}
    }
  \entry
    {2006-2008}
    {Research Assistant - Fudan University}
    {Shanghai, China}
    {Participated in studies on central visual system of mammals. Investigated pattern adaptation in striate cortex of cats via intrinsic optical signal imaging method.}
\end{entrylist}

\section{Education}

\begin{entrylist}
  \entry
    {2008-2010}
    {M.P.H. in Epidemiology - University of Minnesota, Twin Cities}
    {Minneapolis, MN}
    {observational and clinical trial study design, generalized linear model, survival analysis, SAS and R programming, cancer epidemiology}
  \entry
    {2004–2008}
    {B.S. in Biological Science - Fudan University}
    {Shanghai, China}
    {biostatistics, pathophysiology, virology, immunology, genetics, biochemistry}
\end{entrylist}

\clearpage

\section{Publications}

\printbibsection{article}{Articles in peer-reviewed journals}
\printbibsection{inproceedings}{Presentations in peer-reviewed conferences}

\section{Peer-review Activities}

\begin{entrylist}
  \simpleentry
    {2012}
    {Journal reviewer for \textit{Cancer Causes and Control}}
    {}
\end{entrylist}

\section{Other Projects}

\begin{entrylist}
  \simpleentry
    {2015}
    {SAS support in Atom}
    {\href{https://atom.io/packages/language-sas}{https://atom.io/packages/language-sas}}
  \simpleentry
    {2012}
    {VIM indent script for SAS}
    {\href{https://github.com/vim-scripts/SAS-Indent}{https://github.com/vim-scripts/SAS-Indent}}
  \simpleentry
    {2011}
    {VIM syntax script for SAS}
    {\href{https://github.com/vim-scripts/SAS-Syntax}{https://github.com/vim-scripts/SAS-Syntax}}
\end{entrylist}

\end{document}
